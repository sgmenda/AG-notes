\documentclass[english,12pt]{article}

\usepackage[utf8]{inputenc}
\usepackage{amssymb}

\renewcommand{\familydefault}{\sfdefault}

\usepackage{libertine}
\usepackage{libertinust1math}
\usepackage[T1]{fontenc}

\usepackage[margin=1in]{geometry}

\usepackage{float}
\usepackage[colorlinks=true,allcolors=blue]{hyperref}

\usepackage{tikz}
\usetikzlibrary{cd}
\usepackage{diagram}
\usepackage[all]{xypic}

\usepackage{amsthm}
\newtheorem{theorem}{Theorem}
\newtheorem{lemma}[theorem]{Lemma}
\newtheorem{prop}[theorem]{Proposition}
\newtheorem{conjecture}[theorem]{Conjecture}
\theoremstyle{definition}
\newtheorem{definition}[theorem]{Definition}
\newtheorem{example}[theorem]{Example}
\newtheorem{aexample}[theorem]{Abstract Example}
\newtheorem{cexample}[theorem]{Concrete Example}
\newtheorem*{idea}{Idea}
\newtheorem*{note}{Note}

\DeclareMathOperator{\obj}{obj}
\DeclareMathOperator{\Mor}{Mor}

\newcommand{\C}{\mathcal{C}}
\newcommand{\op}{\text{op}}

\setlength{\parskip}{1mm}
\setlength{\parindent}{0mm}

\title{Introduction to Categories}

\author{Sanketh Menda\\
  University of Waterloo}

\date{June 8, 2018}

\begin{document}

\maketitle

Some References:
\begin{enumerate}
\item Mac Lane, Saunders. Categories for the Working Mathematician.
\item Baez, John.
  \href{http://www.math.ucr.edu/home/baez/categories.html}{Categories,
    Quantization, and Much More.}
\item Osborne, M. Scott. Basic Homological Algebra. Chapters 1 and 2
\end{enumerate}

This talk is not going to be on algebraic geometry. It is gonna be on some
abstract nonsense I wish I knew before I started learning algebraic
geometry. 

\begin{quote}\it
  Please feel free to ask questions.
\end{quote}

Also, I will try to be as concrete as possible and I will mention quantum
computing at least once in this talk.

\section{First, Some Sets}

The notion of a \emph{set} is not always sufficient. For instance, one
cannot have a \emph{set of all sets}, due to Russell's paradox.

\subsection{Russell's Paradox}

Let $S$ be the set of all sets. Then
\begin{equation}
  A = \{X \in S: S \not\in X\}
\end{equation}
is also a valid set. Then notice that for any set $X$,
\begin{equation}
  X \in A \iff X \not\in A.
\end{equation}
Taking $X = A$, we get
\begin{equation}
  A \in A \iff A \not\in A,
\end{equation}
which is a contradiction.

\begin{idea}
  Define something more general than a set.
\end{idea}


\subsection{von Neumann-Bernays-Gödel Set Theory}

To subvert such issues, we need a better theory. In this section, we
restrict ourselves to what are called ``material set theories,'' which
means that we think of elements of sets as existing independently of the
set.

\emph{Von Neumann-Bernays-Gödel set theory} (or NBG) is an extension of
\href{https://ncatlab.org/nlab/show/ZFC}{ZFC} that includes \emph{proper
  classes} (defined below).

The objects of NBG are \emph{classes}. If a class $A$ is contained in
another class $B$, then $A$ is called a \emph{set}. A class that is not a
set is called a \emph{proper class}.

For the axioms of NBG, see the
\href{https://ncatlab.org/nlab/show/NBG}{nLab entry for NBG}.

\section{Categories Now}

\begin{quote}
  \textit{If we try to generalize the heck out of the concept of a group,
    keeping associativity as a sacred property, we get the notion of a
    category.}

  \hfill ---John Baez
\end{quote}

Now that we can talk about all sets at once (using the notion of class),
let us define the \emph{category of sets}---which I will denote by
\textsc{Sets}. The \emph{objects} of this \emph{category} are sets and the
\emph{morphisms} between objects are all the functions between these
sets. Now let us define categories more generally. (I promise, this will
get less crazy real soon.)

\begin{definition}
  A category $\C$ consists of a class of \emph{objects} $\obj \C$ and a set
  of \emph{morphisms}. Every morphism has a \emph{source} and a
  \emph{target}. There is a function $\Mor$ which maps each pair $A,B$ of
  objects to a set $\Mor_\C(A,B)$ of morphisms from $A$ to $B$. We also
  have a pairing function called \emph{composition}
  \begin{align}
    \circ: \Mor(B,C) \times \Mor(A,B) &\to \Mor(A,C) \\
    (g,f) &\mapsto gf
  \end{align}
  which is \emph{associative}; that is, if $f \in \Mor(C,D)$, $g \in
  \Mor(B,C)$ and $h \in \Mor(A,B)$, then $(fg)h = f(gh)$. Finally, each set
  $\Mor(A,A)$ has a distinguished element $i_A$ called \emph{identity}
  which satisfies the unit laws; that is, for every $f \in \Mor(A,B)$
  \begin{equation}
    f = fi_A = i_Bf.
  \end{equation}
\end{definition}

The important part about categories is that they are associative. We like
associativity and we care about it!

\begin{note}
  Morphisms are sometimes called \emph{arrows}.
\end{note}

\begin{note}
  What we call $\Mor$ is sometimes called $\text{Hom}$ for a
  \emph{homomorphism}. But this is bad notation and we will not use it
  here.
\end{note}

\begin{table}[H]\label{tab:conc-examples}
  \centering
  \begin{tabular}{ccc}
    & Objects & Morphisms\\\hline
    \textsc{Grp}& groups & homomorphisms\\
    \textsc{Vect}& vector spaces & linear maps\\
     \textsc{Top}& topological spaces & continuous functions\\
    \textsc{Diff} & smooth manifolds & smooth maps\\
    \textsc{Ring} &rings & ring homomorphisms\\
    \textsc{Hilb} &Hilbert spaces & bounded operators\\[2mm]
  \end{tabular}
  \caption{A Few Concrete Examples of Categories}
\end{table}

In all the above examples, the morphisms are a special function, that is
not always the case. For example, a group is a category where we have one
object and all the morphisms have inverses (such morphisms are called
\emph{isomorphisms}.) These morphisms correspond to the elements of the
group. (This way of looking at groups---as you shall see---turns out to be
very useful. We will return to this when we talk about category
representations which are gross generalizations of group
representations. These super abstract objects actually turn out to be super
useful in quantum gravity!)

\begin{aexample}
  Given a category $\C$, define the \emph{opposite category} $\C^\op$ as
  the category with the same objects but with the arrows
  reversed. Precisely, $\obj \C^\op = \obj \C$ and $\Mor_{C^\op}(A,B) =
  \Mor_\C(B,A)$. Also, the composition is reversed, in $C^\op$
  \begin{equation}
    f \circ g = gf.
  \end{equation}
\end{aexample}

There is something that all the concrete examples in Table
\ref{tab:conc-examples} have but the other examples does not. The concrete
examples all look like a category of ``sets with extra structure.'' This
intuition is formalized using the notion of a concrete category. Don't
worry looking at the size of this definition. We will see a simpler
definition once we learn about functors.

\begin{definition}
  A category $\C$ is called a \emph{concrete category} if it comes equipped
  with a function $\sigma$ from $\obj \C$ to sets such that:
  \begin{enumerate}
  \item For any $A \in \obj \C$, $\sigma(A)$ is a set. This is called the
    \emph{underlying set} of $A$.
  \item $\Mor(A,B)$ consists of functions from $\sigma(A)$ to $\sigma(B)$,
    that is, any $f \in \Mor(A,B)$ is a function from $\sigma(A)$ to
    $\sigma(B)$.
  \item Categorical composition is functional composition.
  \item $i_A$ is the identity map on the set $\sigma(A)$
  \end{enumerate}
\end{definition}

The above definition says that a concrete category is kinda like the
category of sets. We will formalize this next time using functors.

\section{The Relation To Quantum Computing I Promised}

This is just a teaser, for the real deal see the nLab page
\href{https://ncatlab.org/nlab/show/finite+quantum+mechanics+in+terms+of+dagger-compact+categories}{finite
  quantum mechanics in terms of dagger-compact categories} and for the ones
with even more free time
\begin{quote}
  John C. Baez. \href{http://math.ucr.edu/home/baez/quantum/}{Quantum Quandaries: A Category-Theoretic Perspective}.
\end{quote}

Recall \textsc{Hilb}, the category of Hilbert spaces, where the objects are
Hilbert spaces and the morphisms are bounded linear operators. This already
looks a lot like finite-dimensional bounded-energy quantum theory. But let
us take a step back.

Lemme start with a question Bob Coecke asked and answers very well in \href{https://arxiv.org/abs/quant-ph/0510032}{Kindergarten Quantum
Mechanics}, what do we really need to do quantum computing? When we apply a
channel to a state we get another state; that is,
\begin{equation}
  \xymatrix{
    \rho \ar[r]^\Phi & \sigma,
  }
\end{equation}
and we can compose such channels
\begin{equation}
  \xymatrix{
    \rho \ar[r]^\Phi & \sigma \ar[r]^\Psi & \xi.
  }
\end{equation}
We also have an identity map. This looks a lot like a category. But this is
not sufficient. We also want to be able to tensor things together,
remember, this is important! That is, if we have states $\rho$ and $\sigma$
we want to be able to create $\rho \otimes \sigma$. For this, we need what
we in the biz call a
\href{https://ncatlab.org/nlab/show/monoidal+category}{monoidal category}.

\vspace{4mm}
I will end here. Next time we will talk about functors and how category
theory comes into play in algebraic geometry.


\end{document}